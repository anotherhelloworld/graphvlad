\documentclass[12pt]{article}
\usepackage{graphicx}
\usepackage{amssymb}

% Russian specicfic
% -------------------------
\usepackage[T2A]{fontenc}
\usepackage[utf8]{inputenc}
\usepackage[russian]{babel}
% -------------------------

\begin{document}

\title{Общая связность сети. Критическое ребро.}

\author{
  Кирпа Вадим
  \and
  Махлярчук Андрей
  \and
  Утин Никита
  \and
  Березкин Аркадий
%  \and
%  Блинов Игорь
}

\maketitle
\thispagestyle{empty}
\newpage

\section{Постановка задачи:}

\paragraph{}
Для графа $G = (V, E, W)$ с множеством вершин $V$,
множеством ребер $W: E \rightarrow \mathbb{R}_+$
найти ребро $e^*$, такое, что при замене
$W(e^*) \rightarrow \gamma W(e^*)$ сумма сетевых
расстояний между всеми узлами минимизируется
(при $\gamma < 1$) или максимизируется (при $\gamma > 1$).
Расчеты привести для графа Владивостока-2012.

\section{Алгоритм}

\paragraph{}
Для нахождения суммы сетевых расстояний в графе использовался
алгоритм Дейкстры. Чтобы ускорить исполнение программы, из каждой
вершины алгоритм Дейкстры запускался в отдельном потоке.

\paragraph{}
Чтобы найти критическое ребро сумма сетевых расстояний считается
для всех подграфов $G_i = (V, E\textbackslash \{e_i \}, W)$. Если сумма сетевых
ребер в текущем подграфе $G_i$ меньше (больше при $\gamma > 1$)
ранее найденой суммы то это ребро сохраняется в качестве претендента
на критическое. В конце работы алгоритма мы получаем критическое ребро.

\section{Реализация}
Отличная

\section{Результаты}
Хорошие

\section{Заключение}
Окончательное и обжалованию не подлежит

\end{document}
